In general the physics search strategy for new phenomena can be outlined in four
steps, the first one is to define an \emph{hypothesis test}, the \emph{null
  hypothesis}, is that no signal is present i.e.~only known SM processes are
present in the SR and the \emph{alternate hypothesis} is that it exists. The
next steps are meant to quantify which one of the hypotheses is favored by
experimental observations. To this end it is necessary to identify the
observables of the experiment, next a \emph{test--statistic} (a function of
these observables) of the known background and to be tested signal is defined to
rank the experiments from the least to the most signal--like. Finally a rule to
quantify the significance of the exclusion or discovery is chosen.

In the monojet analysis the observable of the experiment is the number of events
in the signal regions defined with the criteria illustrated in
Section~\ref{sec:event-selection}, the $\met$ is the test--statistic and a
\gls{cl} of 95\% is chosen for the exclusion. The number of events observed in
the signal region is governed by a Poisson distribution
\begin{equation}
  \label{eq:86}
  P(O|E) = \frac{E^O }{O!} e^{- E}
\end{equation}
where $O$ is the number of observed events and $E$ is the number of expected
events. In absence of signal (background only hypothesis), the expected number
of events is the total number of predicted background events from SM
processes. If some signal is present (signal plus background hypothesis), the
expected number of events is given by the predicted number of background events
plus the predicted number of signal events. In terms of Poisson distribution,
the probability of observing up to $N_\mathrm{\, SR}$ events in the SR in a
signal plus background hypothesis is:
\begin{equation}
  \label{eq:87}
  P(N_\mathrm{\, SR}|\mu_S + B) = \sum^{N_\mathrm{\, SR}}_{k = 0} \frac{(\mu_S + B)^k}{k!}
  e^{- (\mu_S + B)},
\end{equation}
where $\mu_S$ is the \emph{signal strength} and represents the number of
expected signal events. The lower this probability is, the more likely it is
that the tested signal is not present in the SR and the hypothesis of it
explaining some physics process can be excluded. To quantify the significance of
the exclusion a CL is calculated as:
\begin{equation}
  \label{eq:88}
  CL_\mathrm{\, S+B} = 1 - P(N_\mathrm{\, SR}|S+B) = 1 - \alpha
\end{equation}
where typically $\alpha = 0.05$ and a 95\% CL$_\mathrm{\, S+B}$ is quoted. The
maximum number of expected signal events (S$_\mathrm{\, max}$) for which
CL$_\mathrm{\, S+B}$ < 0.95 can be calculated and if the number of expected
signal events ($S$) exceeds this value (S > S$_\mathrm{\, max}$), the model can
be excluded~\cite{PawelThesis} with a confidence level $\alpha$. In the monojet
analysis the CL$_\mathrm{\, S}$ method~\cite{CLsMethod} is used, where:
\begin{equation}
  \label{eq:89}
  CL_\mathrm{\, S} = \frac{CL_\mathrm{\, S+B}}{CL_\mathrm{\, B}}.
\end{equation}
The CL$_\mathrm{\, S}$ method is preferred in high energy physics. It is
designed to address situations where the observed data could fluctuate below the
predicted SM background. This can happen in the case of low statistic, using
CS$_\mathrm{S + B}$ in this case can lead to exclude BSM signals even if the
experiment has no real sensitivity. The CL$_\mathrm{\, S}$ method allows to deal
with such situations and obtain more conservative limits on the signal
hypothesis by normalizing the confidence level obtained in the signal plus
background hypothesis to the confidence level in the background only
hypothesis.
%%% Local Variables:
%%% mode: latex
%%% TeX-master: "../search_for_DM_LED_with_ATLAS"
%%% End:
