A \gls{cr} is a region of the phase space where the looked for signal
contribution is negligible but the event selection is similar to the signal
region where an enhancement of the signal is expected. The main reason to define
such a region is to check the agreement in shape and normalization between MC
and data samples of kinematic and reconstruction quantities. The $V$ + jets,
where $V$ is either a $W$ or a $Z$ vector boson, constitutes the main background
of the monojet analysis. A pure MC estimation of these processes, suffers from
theoretical uncertainties like the limited knowledge of the \glspl{pdf} and
experimental ones related to the JES and luminosity determination. In order to
estimate the contribution of these backgrounds in the SR, a \emph{data driven}
technique was used. The method aims at reducing the systematic uncertainties by
relying on information from data in the CRs rather than on MC simulations. It
can be divided in three major steps:
\begin{itemize}
\item Define CRs to select $V$ + jets event in data.
\item Calculate a conversion factor from MC observed events in the CR to
  background estimates in the SR\@.
\item Apply the conversion factor to the observed events in the CR to obtain the
  estimate number of events from the process in the SR\@.
\end{itemize}
The CRs used to constrain the $V$ + jets backgrounds have an event selection
that differs from the SR only in the lepton veto and the missing transverse
momentum calculation. They are thus orthogonal to the SR and a minimum
contribution from a monojet-like signal is expected.
%%% Local Variables:
%%% mode: latex
%%% TeX-master: "../search_for_DM_LED_with_ATLAS"
%%% End:
