The search for squark pair production with compressed mass spectrum is carried
out in $pp$ collisions using the data collected by the ATLAS experiment during
the 2015 Run II corresponding to a total integrated luminosity of 3.2 \ifb. The
signal region is defined by the following selection criteria:
\begin{enumerate}[A -]
\item The HLT\_xe70 trigger was used in the whole 2015 dataset.
\item In order to assure that the event originated from a $pp$ collision, a
  primary vertex with at least two associated tracks with $\pt > 0.4$~GeV is
  required.
\item Events in which any jet fails the \emph{loose} jet cleaning criteria are
  rejected. This suppress noise from non-collision background.
\item The most energetic jet in the event (the \emph{leading jet}) must have
  $\pt > 250~$GeV and $|\eta| < 2.4$. Moreover, in order to reject beam-induced
  and cosmic particles background, the event is rejected if the leading jet
  fails the \emph{tight} cleaning criteria.
\item In order to suppresses $Z (\rightarrow \ell \bar{\ell})$ + jets and
  $W (\rightarrow \ell \nu)$ background, events with an identified electron of
  $\pt > 20~$GeV or muon of $\pt > 10~$GeV are rejected, see
  \cref{sec:electrons,sec:muons} for the lepton definitions.
\item In order not to overlap with other ATLAS SUSY searches, events with more
  than four jets are rejected, see \cref{sec:jet-veto} for more details on this
  selection.
\item The $\met$ trigger can select multi--jet events in case of a
  mis--reconstructed jet. In these cases the missing transverse momentum points
  in the direction of one of the jets, this background can be suppressed by
  imposing a minimum azimuthal angle separation between the missing transverse
  momentum and any jet of $\Delta \phi (\mathrm{jets}, \met) > 0.4$.
\item A $\met > 250~$GeV requirement is imposed in order to be able to test
  different BSM signals with different sensitivities to the missing energy.
\end{enumerate}
Inclusive (IM1--IM7) and exclusive (EM1--EM6) \glspl{sr} are defined in the
monojet analysis with increasing $\met$ thresholds from 250~GeV to 700~GeV, see
Table~\ref{tab:event_selection} for the exact definition of the signal
regions. These different $\met$ bins are defined in order to address different
BSM signals tested with the monojet signature. In this chapter special emphasis
is placed on the compressed squark--neutralino model which has been studied by
the author of this thesis.
\begin{table}[!th]
  \centering
  % The @{} is to avoid extra horizontal space
  \begin{tabular}{@{}l@{}c@{}c@{}c@{}c@{}c@{}c@{}c}
    \toprule
    \multicolumn{8}{c}{Event Selection Criteria} \\
    \midrule \midrule
    \multicolumn{8}{l}{HLT\_xe70 trigger} \\
    \multicolumn{8}{l}{Primary Vertex} \\
    \multicolumn{8}{l}{$\met > 250~$GeV} \\
    \multicolumn{8}{l}{Leading jet with $\pt > 250~$GeV and $|\eta| < 2.4$} \\
    \multicolumn{8}{l}{At most 4 jets} \\
    \multicolumn{8}{l}{$\Delta \phi (\mathrm{jets}, \met) > 0.4$} \\
    \multicolumn{8}{l}{Jet quality requirements} \\
    \multicolumn{8}{l}{No identified muon with $\pt > 10~$GeV or electron of
    $\pt > 20~$GeV} \\
    \midrule
    Inclusive SRs & IM1 & IM2 & IM3 & IM4 & IM5 & IM6 & IM7 \\
    $\met$~[GeV] & > 250 & > 300 & > 350 & > 400 & > 500 & > 600 & > 700 \\
    \midrule
    Exclusive SRs & EM1 & EM2 & EM3 & EM4 & EM5 & EM6 \\
    $\met$~[GeV] & [250--300] & [300--350] & [350--400] & [400--500] &
    [500--600] & [600--700] \\
    \bottomrule
  \end{tabular}
  \caption{Definition of the signal region.}
  \label{tab:event_selection}
\end{table}
%%% Local Variables:
%%% mode: latex
%%% TeX-master: "../search_for_DM_LED_with_ATLAS"
%%% End:
