\section{The Standard Model}
The \emph{Standard Model} (SM) is a theoretical model which describes the
elementary constituents of matter and their interactions. Up to now, we
discovered four kind of different interactions, the \emph{electromagnetic}, the
\emph{gravitational}, the \emph{strong} and the \emph{electro-weak interaction};
excluding gravity, all of them are described by means of a \emph{quantum field
  gauge theory}.

The Standard Model is the collection of these gauge theories, it is based on the
gauge symmetry group $SU(3)_C \times SU(2)_L \times U(1)_Y$ where $SU(3)_C$ is
the symmetry group of the \emph{Quantum Chromo-Dynamics} (QCD), the ``C''
subscript stand for \emph{color charge} which is the conserved charge in the
strong interaction. The $SU(2)_L$ is the weak isotopic spin group acting on
\emph{left-handed} doublet of fermions while the $U(1)_Y$ group is the
\emph{hypercharge} symmetry group of the \emph{right-handed} fermion
singlets. Together $SU(2)_L \times U(1)_Y$ form the electro-weak symmetry group.

The Standard Model also contains and (sometimes) predicts the existence of
\emph{elementary particles} that interacts between them via the forces mentioned
above. The matter constituents are called \emph{fermions}, the interaction are
mediated by other particles called \emph{gauge bosons}. Fermions are further
categorized into \emph{quark} and \emph{leptons} and are the true fundamental
constituents of matter; the gauge bosons arise by means of symmetry property of
the Standard Model symmetry group.

The existence of all the leptons, quarks and gauge bosons is confirmed by
experimental tests. Among the bosons, the Higgs boson is peculiar because,
unlike the others, it is not associated with any interaction, instead is
postulated as a consequence of the \emph{spontaneously broken symmetry} of the
electroweak sector which is the property, responsible of giving mass to all the
elementary particles and the weak gauge bosons.

\subsection{Electro-Weak Symmetry Group}
\label{sec:electro-weak-symm}

We can now see how to find out the weak interaction symmetry group, to this end,
let us start by writing out the \emph{Hamiltonian}
\begin{equation}
  H_{weak} = \frac{4 G_F}{\sqrt{2}} J_\mu^\dagger J^\mu
\end{equation}
where
\begin{equation}
  \begin{split}
    J_\mu & \equiv J_\mu^{(+)} = \bar{\psi}_{\nu_e} \gamma_\mu \frac{1}{2} (1 -
    \gamma_5) \psi_e \equiv \bar{\nu}_{e_L}
    \gamma_\mu e_L \\
    J_\mu^\dagger & \equiv J_\mu^{(-)} = \bar{\psi}_e \gamma_\mu \frac{1}{2} (1
    - \gamma_5) \psi_{\nu_e} \equiv \bar{e}_L \gamma_\mu \nu_{e_L}
  \end{split}
\end{equation}
to easy the notation, let us write
\begin{equation}
  \chi_L =
  \begin{pmatrix}
    \nu_{e_L} \\ e^-_L
  \end{pmatrix}
  \equiv
  \begin{pmatrix}
    \nu_e \\ e^-
  \end{pmatrix}
\end{equation}
and using the Pauli matrices
\begin{equation}
  \tau_\pm = \frac{1}{2}( \tau_1 \pm i \tau_2)
\end{equation}
we have
\begin{equation}
  \begin{split}
    J_\mu^{(+)} &= \bar{\chi}_L \gamma_\mu \tau_+ \chi_L \\
    J_\mu^{(-)} &= \bar{\chi}_L \gamma_\mu \tau_- \chi_L
  \end{split}
\end{equation}
by introducing a ``neutral'' current
\begin{equation}
  J_\mu^{(3)} = \bar{\chi}_L \gamma_\mu \frac{\tau_3}{2} \chi_L = \frac{1}{2} \bar{\nu}_L \gamma_\mu \nu_L - \frac{1}{2} \bar{e}_L \gamma e_L
\end{equation}
we have a ``triplet'' of currents
\begin{equation}
  \label{eq:1}
  J_\mu^i = \bar{\chi}_L \gamma_\mu \frac{\tau_i}{2} \chi_L.
\end{equation}

Now if we pick up an $SU(2)_L$ transformation
\begin{equation}
  \chi_L (x) \to \chi'_L (x) = e^{i \vec{\varepsilon} \cdotp \vec{T}} \chi_L(x) = e^{i \vec{\varepsilon} \cdotp \frac{\vec{\tau}}{2}} \chi_L(x),
\end{equation}
where $T_i = \tau_i / 2$ are the $SU(2)_L$ \emph{generators}, and think the
$\chi_L$ as the \emph{fundamental representation}, then the current triplet is a
triplet of $SU(2)_L$, the \emph{weak isotopic spin}.

The right handed fermions are singlet for the $SU(2)_{L}$, thus
\begin{equation}
  e_{R} \to e'_{R}= e_{R}.
\end{equation}
Since we are considering the global transformations, we have no interaction, so
the Lagrangian reads
\begin{equation}
  \label{eq:2}
  \mathcal{L} = \bar{e} i \gamma^{\mu} \partial_{\mu} e + \bar{\nu} i
  \gamma^{\mu} \partial_{\mu} \nu \equiv \bar{\chi_{L}} i
  \gamma \partial \chi_{L} + \bar{e}_{R} i \gamma \partial e_{R};
\end{equation}
for now we are bounded to set $m_{e} = 0$, in fact the mass term couples right
and left fermion's components and it is not $SU(2)_{L}$ invariant.  In 1973,
experiments detected events of the type
\begin{equation}
  \label{eq:3}
  \bar{\nu}_{\mu}\eminus \rightarrow \bar{\nu}_{\mu}\eminus \\
\end{equation}
\begin{equation}
  \label{eq:4}
  \begin{cases}
    \nu_{\mu} N \rightarrow \nu_{\mu} X \\
    \bar{\nu}_{\mu} N \rightarrow \bar{\nu}_{\mu} X
  \end{cases}
\end{equation}
which are evidence of a neutral current. Further investigations yielded that the
neutral weak current is predominantly $V-A$ (i.e. left-handed) but not purely
$V-A$ so the $J^{(3)}_{\mu}(x)$ current introduced above can not be used as it
involves only left handed fermions.  We know a neutral current that mixes left
and right components namely the electromagnetic current
\begin{equation}
  \label{eq:5}
  J_{\mu} \equiv e J_{\mu}^{(em)} = e \bar{\psi} \gamma_{\mu} Q \psi
\end{equation}
where $Q$ is the charge operator with eigenvalue $Q = -1$ for the electron. $Q$
is the generator of the $U(1)_{(em)}$ group. So we have an isospin triplet and
we have included the right hand components, the isospin singlet, what we want to
do, is to combine them and define the hypercharge operator
\begin{equation}
  \label{eq:6}
  Y = 2 ( Q - T_{3}) \rightarrow Q = T_{3} + \frac{Y}{2},
\end{equation}
for the current we have
\begin{equation}
  \label{eq:7}
  J_{\mu}^{(em)} = J_{\mu}^{(3)} + \frac{1}{2} J_{\mu}^{Y}
\end{equation}
where
\begin{equation}
  \label{eq:8}
  J_{\mu}^{Y} = \bar{\psi} \gamma_{\mu} Y \psi
\end{equation}
so, by analogy, the hypercharge $Y$ generates a $U(1)_{Y}$ symmetry, and, as it
is a $SU(2)_{L}$ singlet, leaves \eqref{eq:2} invariant under the
transformations
\begin{equation}
  \label{eq:9}
  \begin{split}
    \chi_{L}(x) \rightarrow \chi'_{L}(x) = e^{i \beta Y} \chi_{L}(x)
    \equiv e^{i \beta y_{L}} \chi_{L} \\
    e_{R}(x) \rightarrow e'_{R}(x) = e^{i \beta Y}e_{R}(x) \equiv e^{i \beta
      y_{R}}e_{R}.
  \end{split}
\end{equation}
We thus have incorporated the electromagnetic interactions extending the group
to $SU(2)_{L} \times U(1)_{Y}$ and instead of having a single symmetry group we
have a direct product of groups, each with his own \emph{coupling constant}, so,
in addition to $e$ we will have another coupling to be found.  Since we have a
direct product of symmetry groups, the generators of $SU(2)_{L}$, $T_{i}$, and
the generators of $U(1)_{Y}$, $Y$ commute, the commutation relations are
\begin{equation}
  \label{eq:10}
  [T_{+},T_{-}] = 2 T_{3} \quad ; \quad [T_{3},T_{\pm}] = \pm T_{\pm}
  \quad ; \quad [Y,T_{\pm}] = [Y,T_{3}] = 0,
\end{equation}
member of the same isospin triplet, have same hypercharge eigenvalue; the
relevant quantum numbers are summarized in the table ~\ref{tab:hyper}.
\begin{table}[htb]
  \renewcommand{\arraystretch}{1.25}
  \centering
  \begin{tabular}{l c c c c}
    \hline
    {\bf Lepton} & $T$ & $T^{(3)}$ & $Q$ & $Y$ \\ \hline\hline
    $\nu_{e}$ & $\frac{1}{2}$ & $ \phantom{-}\frac{1}{2}$ & 0 & -1 \\
    $e_{L}^{-}$ & $\frac{1}{2}$ & $-\frac{1}{2}$ & -1 & -1 \\
    \\
    $e_{R}^{+}$ & 0 & $\phantom{-}0$ & -1 & -2 \\ \hline
  \end{tabular} \quad
  \begin{tabular}{l c c c c}
    \hline
    {\bf Quark} & $T$ & $T^{(3)}$ & $Q$ & $Y$ \\ \hline\hline
    $u_{L}$ & $\frac{1}{2}$ & $\phantom{-}\frac{1}{2}$ & $\phantom{-}\frac{2}{3}$ &
                                                                                    $\phantom{-}\frac{1}{3}$ \\
    $d_{L}$ & $\frac{1}{2}$ & $-\frac{1}{2}$ & $-\frac{1}{2}$ &
                                                                $\phantom{-}\frac{1}{3}$ \\
    $u_{R}$ & 0 & $\phantom{-}0$ & $\phantom{-}\frac{2}{3}$ & $\phantom{-}\frac{4}{3}$ \\
    $d_{R}$ & 0 & $\phantom{-}0$  $$ & $-\frac{1}{3}$ & $-\frac{2}{3}$ \\ \hline


  \end{tabular}
  \caption{Weak Isospin and Hypercharge Quantum Numbers of Leptons and Quarks}
  \label{tab:hyper}
\end{table}

\subsection{Electro-Weak Interactions}
\label{sec:electro-weak-inter}
As stated before, interactions are mediated by a gauge boson, we now want to
find out those for the electroweak interaction, to this end let us consider
\emph{local} gauge transformations
\begin{equation}
  \label{eq:11}
  \begin{split}
    \chi_{L} \rightarrow \chi'_{L} &= e^{i \vec\epsilon(x) \cdot \vec T
      + i \beta(x) Y} \chi_{L} \\
    \psi_{R} \to \psi'_{R} &= e^{i \beta(x) Y} \psi_{R},
  \end{split}
\end{equation}
introducing four gauge bosons,
$W_{\mu}^{(1)}, W_{\mu}^{(2)}, W_{\mu}^{(3)}, B_{\mu}$ (same as the number of
generators) and the \emph{covariant derivative}
\begin{equation}
  \label{eq:12}
  \begin{split}
    D_{\mu} \chi_{L} &= (\partial_{\mu} + i g \frac{\vec \tau}{2} \cdot
    \overrightarrow{W}_{\mu}(x) + i \frac{g'}{2} y_{L} B_{\mu}(x)) \chi_{L}
    \\
    &= (\partial_{\mu} + i g \frac{\vec \tau}{2}
    \cdot \overrightarrow{W}_{\mu}(x) - i \frac{g'}{2} B_{\mu}(x)) \chi_{L} \\
    D_{\mu} \psi_{R} &= (\partial_{\mu} + i \frac{g'}{2} y_{R} B_{\mu}(x))
    \psi_{R} \\
    &= (\partial_{\mu} - i \frac{g'}{2} B_{\mu}(x)) e_{R}
  \end{split}
\end{equation}
the Lagrangian \eqref{eq:2} reads
\begin{equation}
  \label{eq:13}
  \begin{split}
    \mathcal{L} &= \bar{\chi}_{L} i \gamma \partial \chi_{L} + \bar{e}_{R} i
    \gamma \partial e_{R} - g \bar{\chi}_{L} \gamma^{\mu} \frac{\vec{\tau}}{2}
    \chi_{L} \overrightarrow{W}_{\mu} + \frac{g'}{2} (\bar{\chi}_{L}
    \gamma^{\mu} \chi_{L} + 2 \bar{e}_{R} \gamma^{\mu} e_{R}) B_{\mu} \\
    &- \frac{1}{4} \overrightarrow{W}_{\mu\nu} \overrightarrow{W}^{\mu\nu} -
    \frac{1}{4} B_{\mu\nu}B^{\mu\nu}
  \end{split}
\end{equation}
where
\begin{equation}
  \label{eq:14}
  \begin{split}
    \overrightarrow{W}_{\mu\nu} &= \partial_{\mu} \overrightarrow{W}_{\nu}
    - \partial_{\nu} \overrightarrow{W}_{\nu}
    - g \overrightarrow{W}_{\mu} \times \overrightarrow{W}_{\nu} \\
    B_{\mu\nu} &= \partial_{\mu} B_{\nu} - \partial_{\nu} B_{\nu}
  \end{split}
\end{equation}
are the kinetic plus non abelian interaction term for the $SU(2)_{L}$ symmetry
(first equation) and the kinetic term for the abelian symmetry group
$U(1)_{Y}$. We can now split the Lagrangian terms to find out the field of the
vector bosons coupled to the charged current and to the neutral current.

\paragraph{Charged Currents Interaction}
\label{sec:charg-curr-inter}
Let us consider the term
\begin{equation}
  \label{eq:15}
  \mathcal{L}_{int}^{ew} = - g \bar{\chi}_{L} \gamma_{\mu}
  \frac{\vect{\tau}}{2} \chi_{L} \vect{W}_{\mu} + \frac{g'}{2}
  \bar{\chi}_{L} \gamma_{\mu} \chi_{L} B^{\mu} + g' \bar{e_{R}}
  \gamma_{\mu} e_{R} B^{\mu}
\end{equation}
defining
\begin{equation}
  \label{eq:16}
  W^{\pm}_{\mu} = \frac{1}{\sqrt{2}} W^{(1)} \mp i W^{(2)}
\end{equation}
we can write
\begin{equation}
  \label{eq:17}
  \mathcal{L}^{CC} = - \frac{g}{\sqrt{2}} (J^{(+)}_{\mu} W^{- \mu} +
  J^{(-)}_{\mu} W^{+ \mu})
\end{equation}
and recognize two charged vector bosons with coupling given by ``$g$''.

\paragraph{Neutral Current Interaction}
\label{sec:neutr-curr-inter}
The relevant term left to consider for what concerns the electroweak currents is
\begin{equation}
  \label{eq:18}
  \mathcal{L}^{NC} = -g J_{\mu}^{(3)} W^{(3) \mu} - \frac{g'}{2}
  J_{\mu}^{Y} B^{\mu},
\end{equation}
the electromagnetic interaction, $-i e J^{(em) \mu} A_{\mu}$, is embedded in
this expression as will became clear considering the \emph{spontaneously broken
  symmetry} phenomena, for now, is sufficient to define
\begin{equation}
  \label{eq:19}
  \begin{split}
    W^{(3)}_{\mu} &= \phantom{-} \cos \theta_{w} Z_{\mu} + \sin \theta_{w}
    A_{\mu}
    \\
    B^{\phantom{(3)}}_{\mu} &= - \sin \theta_{w} Z_{\mu} + \cos \theta_{w}
    A_{\mu}
  \end{split}
\end{equation}
and invert to get
\begin{equation}
  \label{eq:20}
  \begin{split}
    A_{\mu} &= \sin \theta_{w} W_{\mu}^{(3)} + \cos \theta_{w} B_{\mu}
    \\
    Z_{\mu} &= \cos \theta_{w} W_{\mu}^{(3)} - \sin \theta_{w} B_{\mu}
  \end{split}
\end{equation}
where $\theta_{w}$ is the electroweak \emph{mixing angle}. Plugging this into
\eqref{eq:18} and rearranging terms
\begin{equation}
  \label{eq:21}
  \begin{split}
    \mathcal{L}^{NC} &= -[(g \sin \theta_{w} J_{\mu}^{(3)} +
    \frac{g'}{2} \cos \theta_{w} J_{\mu}^{Y} ) A^{\mu} \\
    &\phantom{=} + \phantom{[}(g\cos \theta_{w} J_{\mu}^{(3)} - \frac{g'}{2}
    \sin \theta_{w} J_{\mu}^{Y}) Z^{\mu}]
  \end{split}
\end{equation}
since $A^{\mu}$ is the photon field, the first parenthesis must be identified
with the electromagnetic current, thus
\begin{equation}
  \label{eq:22}
  -(g \sin \theta_{w} J_{\mu}^{(3)} + \frac{g'}{2} \cos
  \theta_{w} J_{\mu}^{Y} ) A^{\mu} = - e J_{\mu}^{(em)} A^{\mu}
  \equiv - e ( J_{\mu}^{(3)} + \frac{J_{\mu}^{Y}}{2} ) A^{\mu}
\end{equation}
from which we get the relation
\begin{equation}
  \label{eq:23}
  g \sin \theta_{w} = g' \cos \theta_{w} = e
\end{equation}
and so we can rewrite \eqref{eq:18},
\begin{equation}
  \label{eq:24}
  \mathcal{L}^{NC} = - \frac{g}{\cos \theta_{w}} [J_{\mu}^{(3)} - \sin^{2} \theta_{w}
  J_{\mu}^{(em)}] Z^{\mu}
\end{equation}
so that $Z^{\mu}$ can be identified with the field for the neutral vector boson.

%%% Local Variables:
%%% mode: latex
%%% TeX-master: "../search_for_DM_LED_with_ATLAS"
%%% End:
