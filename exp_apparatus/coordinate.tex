A right handed coordinate system is defined for the ATLAS detector, the origin
is at the nominal interaction point with the $z$--axis oriented along the beam
direction and the $xy$ plane orthogonal to it. The positive $x$-axis starts at
the interaction point and goes to the center of the LHC ring while the positive
$y$-axis is defined as pointing upwards. The A-side of the detector is defined
as that with a positive $z$-axis while the C-side has the negative $z$-axis.

The LHC beam are unpolarized and thus invariant under rotations around the beam
line axis, a cylindrical coordinate system is particularly convenient to
describe the detector geometry where:
\begin{equation}
  \label{eq:57}
  r = \sqrt{x^2 + y^2}, \quad \phi = \arctan \frac{y}{x}.
\end{equation}
A momentum dependent coordinate, the \emph{rapidity}, is commonly used in
particle physics for its invariance under Lorentz transformations. The rapidity
is defined as:
\begin{equation}
  \label{eq:58}
  y = \frac{1}{2} \ln \frac{E + p_z}{E - p_z}
\end{equation}
where $E$ is the energy of the particle and $p_z$ its momentum along the
$z$-axis. Rapidity intervals are Lorentz invariant. In the relativistic limit or
when the mass of the particle is negligible, the rapidity only depends on the
production angle of the particle with respect to the beam axis,
\begin{equation}
  \label{eq:59}
  \theta = \arctan \frac{\sqrt{p_x^2 + p_y^2}}{p_z}.
\end{equation}
This approximation is called \gls{eta} and is defined as:
\begin{equation}
  \label{eq:60}
  y \xrightarrow{p \gg m} \eta = - \ln \left( \tan \frac{\theta}{2} \right).
\end{equation}
A value of $\theta = 90^{\circ}$, perpendicular to the beam axis, corresponds to
$\eta = 0$. The spatial separation between particles in the detector is commonly
given in terms of a Lorentz invariant variable defined as:
\begin{equation}
  \label{eq:61}
  \Delta R = \sqrt{\Delta \phi^2 + \Delta \eta^2}.
\end{equation}

Other quantities used to describe the kinematics of the $pp$ interaction are the
\gls{pt} and the \gls{et} defined as $\pt = p \sin \theta$ and
$\et = E \sin \theta$ respectively.
%%% Local Variables:
%%% mode: latex
%%% TeX-master: "../search_for_DM_LED_with_ATLAS"
%%% End:
