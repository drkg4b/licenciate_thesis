The inner detector (ID) is designed to provide good track reconstruction,
precise momentum resolution and both primary and secondary vertex measurements
above a nominal $\pt$ threshold of 0.5~GeV and within the pseudorapidity
$|\eta| < 2.5$. It also provides electron identification over $|\eta| < 2.0$ for
energies between 0.5~GeV and 150~GeV\cite{ATLASPaper}. The ID is 6.2~m long and
has a radius of about 1.1~m, it is surrounded by a solenoidal magnetic field of
2~T. Its layout is schematized in Figure~\ref{fig:id} and, as can be seen, it is
composed of three sub-detectors.

At the inner radius the \emph{pixel detector} mostly determines the position of
primary and secondary vertex. The silicon sensors are 250~$\mu$m thick detectors
that operate with an initial bias voltage of $\sim$150~V that, due to the high
radiation level, will increase up to 600~V after 10 years of operation to
maintain a good charge collection.

In the middle layer of the ID the \emph{semiconductor tracker} (SCT) is designed
to give eight precision measurements per track which contributes to determine
the primary and secondary vertex position and momentum measurements. The silicon
sensors are 285 $\pm 15 \mu$m thick and initially operates with a bias voltage
of $\sim$150~V which will increase up to 350~V after ten years of operation for
good charge collection.

The last layer of the ID is the \emph{transition radiation tracker} (TRT), it
contributes to tracking and identification of charged particles. It consists of
drift (straw) tubes, 4~mm in diameter with a 31~$\mu$m wire in the center of
each straw, filled with a gas mixture of 70\% Xe, 27\% CO$_2$ and 3\%
O$_2$. These tubes substantially act like proportional counters where the tube
is the cathode and kept at $- 1.5$~kV and the wire is the anode and
grounded. When a charged particle cross one tube, leaves a signal; the set of
signals in the tubes, reconstructs to a track which represents the path of the
crossing object. The space between the straw tubes is filled with material with
different refraction index, this causes charged particles crossing it to emit
transition radiation thus leading to some straw to have a much stronger
signal. The transition radiation depends on the speed of the particles which in
turn depends on the initial energy and the mass of the particles thus lighter
particles will have higher transition energy and stronger signal in the straw
tubes. Tracks with several strong signal straw, can be identified as belonging
to electrons (the lightest charged particle).

\begin{figure}[!h]
  \centering
    \includegraphics[width=.5\linewidth]{inner_detector}
    \caption{Schematic view of a charged track of 10~GeV $\pt$ that traverses
      the different ID sub-detectors. After traversing the beryllium pipe, the
      track passes through the three cylindrical silicon-pixel layers, the four
      layers of silicon-microstrip sensors (SCT) and the approximately 36 straws
      contained in the TRT within their support structure.}
    \label{fig:id}
\end{figure}
%%% Local Variables:
%%% mode: latex
%%% TeX-master: "../search_for_DM_LED_with_ATLAS"
%%% End:
