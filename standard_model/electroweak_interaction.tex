As stated before, interactions are mediated by a gauge boson, we now want to
find out those for the electroweak interaction, to this end let us consider
\emph{local} gauge transformations
\begin{equation}
  \label{eq:11}
  \begin{split}
    \chi_{L} \rightarrow \chi'_{L} &= e^{i \vec\epsilon(x) \cdot \vec T
      + i \beta(x) Y} \chi_{L} \\
    \psi_{R} \to \psi'_{R} &= e^{i \beta(x) Y} \psi_{R},
  \end{split}
\end{equation}
introducing four gauge bosons,
$W_{\mu}^{(1)}, W_{\mu}^{(2)}, W_{\mu}^{(3)}, B_{\mu}$ (same as the number of
generators) and the \emph{covariant derivative}
\begin{equation}
  \label{eq:12}
  \begin{split}
    D_{\mu} \chi_{L} &= (\partial_{\mu} + i g \frac{\vec \tau}{2} \cdot
    \overrightarrow{W}_{\mu}(x) + i \frac{g'}{2} y_{L} B_{\mu}(x)) \chi_{L}
    \\
    &= (\partial_{\mu} + i g \frac{\vec \tau}{2}
    \cdot \overrightarrow{W}_{\mu}(x) - i \frac{g'}{2} B_{\mu}(x)) \chi_{L} \\
    D_{\mu} \psi_{R} &= (\partial_{\mu} + i \frac{g'}{2} y_{R} B_{\mu}(x))
    \psi_{R} \\
    &= (\partial_{\mu} - i \frac{g'}{2} B_{\mu}(x)) e_{R}
  \end{split}
\end{equation}
the Lagrangian \eqref{eq:2} reads
\begin{equation}
  \label{eq:13}
  \begin{split}
    \mathcal{L} &= \bar{\chi}_{L} i \gamma \partial \chi_{L} + \bar{e}_{R} i
    \gamma \partial e_{R} - g \bar{\chi}_{L} \gamma^{\mu} \frac{\vec{\tau}}{2}
    \chi_{L} \overrightarrow{W}_{\mu} + \frac{g'}{2} (\bar{\chi}_{L}
    \gamma^{\mu} \chi_{L} + 2 \bar{e}_{R} \gamma^{\mu} e_{R}) B_{\mu} \\
    &- \frac{1}{4} \overrightarrow{W}_{\mu\nu} \overrightarrow{W}^{\mu\nu} -
    \frac{1}{4} B_{\mu\nu}B^{\mu\nu}
  \end{split}
\end{equation}
where
\begin{equation}
  \label{eq:14}
  \begin{split}
    \overrightarrow{W}_{\mu\nu} &= \partial_{\mu} \overrightarrow{W}_{\nu}
    - \partial_{\nu} \overrightarrow{W}_{\nu}
    - g \overrightarrow{W}_{\mu} \times \overrightarrow{W}_{\nu} \\
    B_{\mu\nu} &= \partial_{\mu} B_{\nu} - \partial_{\nu} B_{\nu}
  \end{split}
\end{equation}
are the kinetic plus non abelian interaction term for the $SU(2)_{L}$ symmetry
(first equation) and the kinetic term for the abelian symmetry group
$U(1)_{Y}$. We can now split the Lagrangian terms to find out the field of the
vector bosons coupled to the charged current and to the neutral current.
%%% Local Variables:
%%% mode: latex
%%% TeX-master: "../search_for_DM_LED_with_ATLAS"
%%% End:
