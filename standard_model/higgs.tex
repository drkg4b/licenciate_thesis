Up to now, we have massless gauge vector bosons, in fact no term such as
$M^{2} B_{\mu}B^{\mu} / 2$ appear in the Lagrangian \eqref{eq:13}, but this kind
of terms are not gauge invariant and thus we can not just add them or we will
end up with troubles later when trying to renormalize the theory.

A gauge invariant way to recover the fermions and bosons masses, is to
spontaneously brake the local $SU(2)_{L} \times U(1)_{Y}$ electroweak symmetry.

\subsection{Non Abelian Spontaneously Broken Symmetry}
\label{sec:non-abel-spont}
Let us consider a local symmetry breaking and refer to~\cite{martin:particle}
for a more complete explanation. Be $\phi$ a complex scalar field,
\begin{equation}
  \label{eq:25}
  \mathcal{L} = (\partial_{\mu} \phi^{*})(\partial_{\mu} \phi) -
  \underbrace{\mu^{2}\phi^{*}\phi - \lambda(\phi^{*}\phi)^{2}}_{V(\phi^{*}\phi)}
\end{equation}
setting
\begin{equation}
  \label{eq:26}
  \begin{split}
    \phi^{\phantom{*}} &= \frac{\phi_{1} + i \phi_{2}}{\sqrt{2}} \\
    \phi^{*} &= \frac{\phi_{1} - i \phi_{2}}{\sqrt{2}}
  \end{split}
\end{equation}
we get
\begin{equation}
  \label{eq:27}
  \mathcal{L} = \frac{1}{2} (\partial_{\mu} \phi_{1})^{2} +
  \frac{1}{2} ( \partial_{\mu} \phi_{2} )^{2} - \frac{\mu^{2}}{2} (
  \phi_{1}^{2} + \phi_{2}^{2} ) - \frac{\lambda}{4} ( \phi_{1}^{2} +
  \phi_{2}^{2} )^{2}
\end{equation}
the gauge transformations are
\begin{equation}
  \label{eq:28}
  \begin{cases}
    \phi^{\phantom{\dagger}}(x) \rightarrow \phi^{\phantom{\dagger}'}(x) =
    e^{\phantom{-} i \epsilon}
    \phi^{\phantom{\dagger}}(x) \\
    \phi^{\dagger} (x) \rightarrow \phi^{\dagger '} (x) = e^{- i \epsilon}
    \phi^{\dagger}(x).
  \end{cases}
\end{equation}
There are two possible choices for the potential
\begin{itemize}
\item[-] $\mu^{2} > 0$, which gives a stable configuration around $|\phi| = 0$.
\item[-] $\mu^{2} < 0$, which gives a circle of minima such that
  $\phi_{1}^{2} + \phi_{2}^{2} = v^{2}$, with $v^{2} = - \mu^{2} /
  \lambda$. This minima are not gauge invariant, in fact
  \begin{equation}
    \label{eq:29}
    \phi_{0} = \bra 0 \phi \ket 0 \rightarrow \frac{v}{\sqrt{2}} e^{i
      \alpha} \quad \mathtt{if} \quad \phi \rightarrow e^{i \alpha} \phi
  \end{equation}
\end{itemize}
To get the particle interaction we make a perturbative expansion around one
minimum, we chose one, for example $\alpha = 0$, for which $\phi_{1} = v$ and
$\phi_{2} = 0$ and introduce the two perturbations $\eta(x)$ and $\xi(x)$ so
that
\begin{equation}
  \label{eq:30}
  \phi (x) = \frac{1}{\sqrt{2}} \overbrace{v + \xi(x)}^{\phi_{1}} + i
  \overbrace{\eta(x)}^{\phi_{2}}
\end{equation}
and plug them in the Lagrangian \eqref{eq:27} to obtain
\begin{equation}
  \label{eq:31}
  \begin{split}
    \mathcal{L}' (\xi,\eta) &= \frac{1}{2}(\partial_{\mu} \xi)^{2} + \frac{1}{2}
    (\partial_{\mu} \eta)^{2} - \frac{1}{2}(-2 \mu^{2})\eta^{2} \\ &- \lambda v
    (\eta^{2} + \xi^{2}) \eta - \frac{1}{4}(\eta^{2} + \xi^{2})^{4} + \cdots
  \end{split}
\end{equation}
as we can see, the third term looks like a mass term so that the field $\eta$
has mass $m_{\eta}^{2} = -2 \mu^{2}$ while we have no mass term for the field
$\xi$.

This ``trick'' to give mass to one of the gauge field, is the \emph{braking of
  the symmetry}. In fact, by choosing one particular vacuum among the infinite
ones, we lost our gauge invariance; moreover, we ended up with a scalar gauge
boson, known as \emph{Goldstone boson}. We need to find a way to recover the
masses of the gauge bosons in a gauge invariant way by getting rid of massless
scalar fields; the solution is the topic of the very next section.  next
section.

\subsection{The Higgs Mechanism}
\label{sec:higgs-model}
Consider now a local gauge $SU(2)$ symmetry, the field transformations
are
\begin{equation}
  \label{eq:33}
  \phi(x) \rightarrow \phi'(x) = e^{i \textstyle{\sum_{k = 1}^{3}}
    \epsilon^{k} T^{k}} \phi(x),
\end{equation}
where $T^{k} = \frac{\tau^{k}}{2}$ and $[T^{i},T^{j}] = i
\epsilon^{ijk}T^{k}$ with $i,j,k = 1,2,3$. To achieve invariance for
the Lagrangian
\begin{equation}
  \label{eq:34}
  \mathcal{L} =(\partial_{\mu} \phi)^{\dagger}(\partial^{\mu} \phi) -
  \mu^{2}\phi^{\dagger}\phi - \lambda (\phi^{\dagger}\phi)^{2},
\end{equation}
where
\begin{equation}
  \label{eq:35}
  \phi \equiv \binom{\phi_{i}}{\phi_{j}} = \frac{1}{\sqrt{2}}
  \binom{\phi_{1} + i \phi_{2}}{\phi_{3} + i \phi_{4}},
\end{equation}
we need to introduce the covariant derivative
\begin{equation}
  \label{eq:36}
  D_{\mu} = \partial_{\mu} + i g\,\tfrac{\vec \tau}{2} \cdot \vect{W}_{\mu}(x).
\end{equation}
In the case of infinitesimal transformations, the fields transform
like
\begin{equation}
  \label{eq:37}
  \phi(x) \rightarrow \phi'(x) \simeq ( 1 + i \vec \epsilon\, (x) \cdot
  \tfrac{\vec \tau}{2}) \phi(x)
\end{equation}
while the gauge bosons transformations are
\begin{equation}
  \label{eq:38}
  \vect{W}_{\mu} (x) \rightarrow \vect{W}_{\mu}(x) -
  \frac{1}{g} \partial_{\mu} \vec \epsilon \, (x) - \vec \epsilon\,(x)
  \times \vect{W}_{\mu} (x).
\end{equation}
Replacing everything in the Lagrangian we obtain
\begin{equation}
  \label{eq:39}
  \mathcal{L} = ( \partial_{\mu} \phi + i g\,\tfrac{\vec \tau}{2}
  \cdot \vect{W}_{\mu} \phi)^{\dagger} (\partial_{\mu} \phi + i
  g\,\tfrac{\vec \tau}{2} \cdot \vect{W}_{\mu}) - V(\phi) -
  \frac{1}{4} \vect{W}_{\mu\nu} \cdot \vect{W}^{\mu\nu},
\end{equation}
where the potential is given by
\begin{equation}
  \label{Esq:40}
  V(\phi) = \mu^{2} \phi^{\dagger} \phi + \lambda(\phi^{\dagger} \phi)^{2}
\end{equation}
and the kinetic term is
\begin{equation}
  \label{eq:41}
  \vect{W}_{\mu\nu} = \partial_{\mu} \vect{W}_{\nu} - \partial_{\nu} \vect{W}_{\mu}
  - g\,\vect{W}_{\mu} \times \vect{W}_{\nu}.
\end{equation}

We are interested in the case of the spontaneously broken symmetry,
thus $\mu^{2} < 0$ and $\lambda > 0$. The minima of the potential lie
on
\begin{equation}
  \label{eq:42}
  \phi^{\dagger}\,\phi = \frac{1}{2} (\phi_{1}^{2} + \phi_{2}^{2} +
  \phi_{3}^{2} + \phi_{4}^{2}) = - \frac{\mu^{2}}{2 \lambda}
\end{equation}
and we have to choose one of them, let it be
\begin{equation}
  \label{eq:44}
  \phi_{1} = \phi_{2} = \phi_{4} = 0, \quad \phi_{3}^{2} = -
  \frac{\mu^{2}}{\lambda} \equiv v^{2}.
\end{equation}
To expand $\phi$ around this particular vacuum
\begin{equation}
  \label{eq:43}
  \phi_{0} \equiv \frac{1}{\sqrt{2}} \binom{0}{v}
\end{equation}
it is sufficient to substitute the expansion
\begin{equation}
  \label{eq:45}
  \phi (x) = \frac{1}{\sqrt{2}} \binom{0}{v + h(x)}
\end{equation}
in the Lagrangian \eqref{eq:39} in order to get rid of the,
unobserved, Goldstone bosons and retain only one neutral scalar field,
the \emph{Higgs field}.

% To generate the masses for the three gauge bosons one can simply
% substitute the vacuum expectation value, indeed
% \begin{equation}
%   \label{eq:46}
%   \begin{split}
%     &( i g\, \tfrac{\vec \tau}{2} \cdot \vect{W}_{\mu}
%     \phi_{0})^{\dagger} ( i g\, \tfrac{\vec \tau}{2} \cdot
%     \vect{W}_{\mu}
%     \phi_{0}) =  \\
%     &= \frac{g^{2}}{2}
%     \begin{pmatrix}
%       0 & v
%     \end{pmatrix}
%     \begin{pmatrix}
%       W^{3} & W^{1} - i W^{2} \\
%       W^{1} + i W^{2} & - W^{3}
%     \end{pmatrix}
%     \begin{pmatrix}
%       W^{3} & W^{1} - i W^{2} \\
%       W^{1} + i W^{2} & - W^{3}
%     \end{pmatrix}
%     \binom{0}{v} = \\
%     &= \frac{g^{2}v^{2}}{8} ( W^{1} + i W^{2} ,\; - W^{3})
%     \binom{W^{1} -
%       i W^{2}}{- W^{3}} = \\
%     &= \frac{g^{2} v^{2}}{8} \left[(W_{\mu}^{1})^{2} +
%       (W_{\mu}^{2})^{2} + (W_{\mu}^{3})^{2}\right]
%   \end{split}
% \end{equation}
% and thus we see three vector gauge bosons with mass $M =
% \frac{gv}{2}$. To summarize, the Lagrangian \eqref{eq:39} describes
% three massive gauge fields and a scalar field, the Higgs field.

\subsection{Masses for the $W^{\pm}$ and $Z^{0}$ Gauge Bosons}
\label{sec:masses-wpm-z}
% Consider the Higgs Lagrangian
% \begin{equation}
%   \label{eq:32}
%   \mathcal{L}_{\phi} = [(\partial_{\mu} + i g\, \vect{T} \cdot
%   \vect{W}_{\mu} + i \tfrac{g'}{2}\,Y B_{\mu})\phi]^{\dagger}  [(\partial_{\mu} + i g\, \vect{T} \cdot
%   \vect{W}_{\mu} + i \tfrac{g'}{2}\,Y B_{\mu})\phi] - V(\phi)
%  \end{equation}
%  with $V(\phi) = \mu^{2}\phi^{\dagger} \phi + \lambda (
%  \phi^{\dagger} \phi )^{2}$ and $\lambda > 0$. To preserve
%  $SU(2)_{L} \times U(1)_{Y}$ gauge invariance for this Lagrangian,
%  we can choose four fields to be arranged in an isospin doublet with
%  $Y = 1$
%  \begin{equation}
%   \label{eq:40}
% \phi = \binom{\phi^{+}}{\phi^{0}} \quad \mbox{with}
%   \begin{aligned}
%     &\quad \phi^{+} \equiv
%     \frac{1}{\sqrt{2}} ( \phi_{1} + i \phi_{2}) \\
%     &\quad \phi^{0} \equiv \frac{1}{\sqrt{2}} ( \phi_{3} + i
%     \phi_{4})
%   \end{aligned}
% \end{equation}
% and break the symmetry choosing
% \begin{equation}
%   \label{eq:47}
%   \phi_{0} \equiv \frac{1}{\sqrt{2}} \binom{0}{v}
% \end{equation}
% and taking $\mu^{2} < 0$ and $\lambda > 0$ in the potential
% $V(\phi)$.

The gauge bosons masses are generated simply substituting the vacuum
expectation value, $\phi_{0}$, in the Lagrangian, the relevant term is
\begin{equation}
  \label{eq:48}
  \begin{split}
    \left| ( g\,\tfrac{\vec \tau}{2} \cdot \vect{W}_{\mu} +
      \tfrac{g'}{2}
      B_{\mu}) \phi \right|^{2} &= \\
    &= \frac{1}{8} \left|
      \begin{pmatrix}
        g W^{3}_{\mu} + g' B_{\mu} & g(W^{1}_{\mu} - i W^{2}_{\mu}) \\
        g(W^{1}_{\mu} + iW^{2}_{\mu}) & -g W^{3}_{\mu} + g' B_{\mu}
      \end{pmatrix}
      \binom{0}{v} \right|^{2} \\
    &= \frac{1}{8} v^{2}g^{2} [(W^{1}_{\mu})^{2} + (W^{2}_{\mu})^{2}]
    + \frac{1}{8} v^{2}(g' B_{\mu} - g W_{\mu}^{3})( g' B^{\mu} - g
    W^{3
      \mu}) \\
    &= (\frac{1}{2} gv)^{2} W^{+}_{\mu}W^{- \mu} + \frac{1}{8} v^{2}
    \begin{pmatrix}
      W_{\mu}^{3} & B_{\mu}
    \end{pmatrix}
    \begin{pmatrix}
      g^{2} & - g g' \\
      - g g' & g^{2}
    \end{pmatrix}
    \begin{pmatrix}
      W^{3 \mu} \\ B^{\mu},
    \end{pmatrix}
  \end{split}
\end{equation}
having used $W^{\pm} = ( W^{1} \mp i W^{2} ) / \sqrt{2}$. The mass
term, lead us to conclude that
\begin{equation}
  \label{eq:49}
  M_{W} = \frac{1}{2} g v.
\end{equation}
The remaining term is off diagonal
\begin{equation}
  \label{eq:50}
  \begin{split}
    \frac{1}{8} v^{2} [g^{2} (W_{\mu}^{3})^{2} - 2 g g' W_{\mu}^{3}
    B^{\mu} + g'^{2} B_{\mu}^{2} ] &= \frac{1}{8} v^{2} [g
    W^{3}_{\mu} - g B_{\mu} ]^{2} \\
    \quad &+ 0 \phantom{v^{2}} [g' W^{3}_{\mu} - g' B_{\mu} ]^{2}
  \end{split}
\end{equation}
but one can diagonalize and find that
\begin{equation}
  \label{eq:51}
  \begin{split}
    A^{\mu} &= \frac{g' W_{\mu}^{3} + g B_{\mu}}{\sqrt{g^{2} +
        g'^{2}}} \\
    Z^{\mu} &= \frac{g W_{\mu}^{3} + g' B_{\mu}}{\sqrt{g^{2} +
        g'^{2}}}
  \end{split}
\end{equation}
with $M_{A} = 0$ and $M_{Z} = v \sqrt{g^{2} + g'^{2}} / 2$ which are
the photon and neutral weak vector boson fields.
% Recalling that
% \begin{equation}
%   \label{eq:52}
%   \frac{g'}{g} = \tan \theta_{W}
% \end{equation}
% it is possible to write
% \begin{equation}
%   \label{eq:53}
%   \begin{split}
%     A_{\mu} &= \phantom{-} \cos \theta_{W} B_{\mu} + \sin \theta_{W}
%     W^{3}_{\mu}
%     \\
%     Z_{\mu} &= - \sin \theta_{W} B_{\mu} + \cos \theta_{W}
%     W^{3}_{\mu}.
%   \end{split}
% \end{equation}
Thus the mass eigenstates are a massless vector boson, $A_{\mu}$ and a
massive gauge boson $Z_{\mu}$.

We have shown in this section how the Higgs mechanism can be applied
to give mass to the gauge bosons of the electroweak model.
%%% Local Variables:
%%% mode: latex
%%% TeX-master: "../search_for_DM_LED_with_ATLAS"
%%% End:
