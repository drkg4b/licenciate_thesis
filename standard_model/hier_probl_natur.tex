The \emph{naturalness criterion} states that one such [dimensionless and
measured in units of the cut-off] parameter is allowed to be much smaller than
unity only if setting it to zero increases the symmetry of the theory. If this
does not happen, the theory is unnatural~\cite{thooft:gauge}.

There are two important concepts in physics that enter in the formulation of the
naturalness principle, symmetries and effective field
theories. \emph{Symmetries} are closely connected to conservation laws, moreover
theory parameters that are protected by a symmetry, if smaller than the unit,
are not problematic according to the naturalness criterion. \emph{Effective
  field theories} are a sort of simplification of a more general theory that use
less parameters to describe the dynamics of particles with energies less than a
cut-off scale $\Lambda$.

Let us now consider the strength of the gravitational force, characterized by
the Newton's constant, G$_N$ and the weak force, characterized by the Fermi's
constant G$_F$, if we take the ratio of these we get:
\begin{equation}
  \label{eq:gf_gn_ratio}
  \frac{G_F \hbar^2}{G_N c^2} = 1.738 \times 10^{33}.
\end{equation}
The reason why this number is worth some attention is that theory parameters
close to the order of the unit in the SM, may be calculated in a more
fundamental theory, if any, using fundamental constants like $\pi$ or $e$ while
very big numbers may not have such a simple mathematical expression and thus may
lead to uncover new properties of the fundamental theory.

This number becomes even more interesting if we consider quantum effects.
\emph{Virtual particles} are not really particles but rather disturbances in a
field, these disturbances are off-shell ($E \neq m^2 + p^2$) and according to
the \emph{uncertainty principle}, $\Delta t \Delta E \geq \hbar / 2$, can appear
out of nothing for a short time that depends on the energy of the virtual
particle; according to quantum field theory, the vacuum is populated with such
disturbances. The Higgs field, has the property to couple with other SM
particles with a strength proportional to their mass. Now all these virtual
particles have a mass determined by the available energy $\Lambda$ and when the
Higgs field travels through space, it couples with these virtual particles and,
due to quantum corrections, its motion is affected and its invariant mass
squared gets a contribution proportional to $\Lambda$:
\begin{equation}
  \label{eq:delta_mh}
  \delta m_H^2 = k \Lambda^2 \text{, with } k = \frac{3 G_F}{4 \sqrt{2}
    \pi^2}(4m_t^2 - 2m_W^2 - m_Z^2 - m_H^2).
\end{equation}
Since $k \approx 10^{-2}$\cite{Giudice:2008bi}, the value of Higgs' mass
$m_H \sim G_F^{-1/2}$, should be close to the maximum energy scale $\Lambda$ and
if we assume this to be the Plank scale $M_{Pl} = G_N^{-1/2}$, the ration
$G_F/G_N$, should be close to the unity which contradicts
eq.~\eqref{eq:gf_gn_ratio}, this goes by the name of \emph{hierarchy problem}.

The large quantum corrections in~\eqref{eq:delta_mh} are mainly due to the fact
that in the SM, there is no symmetry protecting the mass of the Higgs'
field. Supersymmetry (SUSY), among other things, is capable of solving the
hierarchy problem by canceling out the quantum corrections that bring $m_H$
close to $\Lambda$ thus restoring the naturalness of the SM.
%%% Local Variables:
%%% mode: latex
%%% TeX-master: "../search_for_DM_LED_with_ATLAS"
%%% End:
