The ATLAS detector is a multipurpose cylindrical detector used in the search for
the Higgs boson, supersymmetry and dark matter constituents. It is 44~m long and
25~m in diameter, it is divided into three main components, the inner detector,
used to track charged particles and momentum measurement, the calorimeters which
measure the energy and the muon spectrometer which provides tracking and
momentum information for the muons. The Tile Calorimeter is the hadronic
calorimeter covering the most central region of the ATLAS detector, it is used
in the measurement of hadrons, jets, taus and the missing energy. To this end,
an understanding of the electronic noise inside the detector is crucial as it
affects the signal left by the particles crossing the calorimeter and must be
therefore monitored and kept up to date. Part of the work presented in this
thesis was to update, monitor and study the noise calibration constants for the
Tile Calorimeter to allow for processing of data and the identification of
hadronic jets. During these studies an unexpected situation was encountered
where an unexpected variation in the cell noise was observed. Further
investigation led to discover that the tile noise filter, an algorithm used to
mitigate the effect of coherent noise, was not behaving as expected in some
situations affecting approximately 5\% of the cells in the TileCal.

The results of the search for new physics phenomena in a monojet with large
missing transverse momentum using the data from $pp$ collisions at LHC collected
by the ATLAS experiment during 2015 and corresponding to an integrated
luminosity of 3.2~$\ifb$ have been presented. No significant excess in the data
compared to the SM predictions has been found thus the results have been
translated into model independent upper limits on the visible cross section with
a 95\% CL and interpreted in terms of squark pair production with
$\squarkprod$. Squark masses up to 608~GeV are excluded.

Naturalness suggests that supersymmetric standard model partners are expected at
the TeV scale, with the high luminosity foreseen in 2016 it will be possible to
test experimentally the squark pair production with masses at that scale.
%%% Local Variables:
%%% mode: latex
%%% TeX-master: "../search_for_DM_LED_with_ATLAS"
%%% End:
