My contribution to the ATLAS experiment started in early 2013 by studying the
electronic noise in the hadronic calorimeter. The Tile Calorimeter is designed
to measure jets, tau particles, missing energy and for the energy reconstruction
of hadrons, thus having an up--to--date description of the noise in the detector
is important for several physics analysis. During the reprocessing of the 2011
data a set of python scripts was developed in order to plot the noise constants
variation over several calibration runs and some unusual variation in the noise
constants was spotted and investigated.

\mbox{}

I later started analyzing the ATLAS data joining the monojet team in the effort
of trying to answer relevant Standard Model open questions. During Run~I my
contribution was limited to the study of some of the systematic
uncertainties. In Run~II the ATLAS software framework was changed thus the
analysis code was updated. Compared to the Run~I analysis an upper cut on the
number of jets was introduced in order to not overlap with other analyses, the
efficiency of such a cut was studied using a multivariate method based on the
TMVA package and the results, being in agreement with other studies performed in
the group, implemented in the analysis. The expected limits on the SUSY
compressed spectra scenario with $\squarkprod$ were estimated and lately
compared to the observed limit in the 3.2~$\ifb$ collected by ATLAS during
Run~II data taking.
%%% Local Variables:
%%% mode: latex
%%% TeX-master: "../search_for_DM_LED_with_ATLAS"
%%% End:
