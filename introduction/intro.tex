The Standard Model of particle physics is the theory used to describe the
elementary constituents of matter and their interactions. Through the years is
has been tested by many experiments and despite its success it cannot explain,
among other problems, the so called hierarchy and dark matter
problems. Supersymmetry is an extension of the Standard Model that could solve
these issues by introducing new particles. The lightest of these particles, in
the context of a minimal supersymmetric model, could be produced through the
$\squarkprod$ process and, lacking electromagnetic and strong
interaction~\cite{MSSMIntro}, escape detection. With an energy in the center of
mass of $\sqrt{s} = 13$~TeV, the large hadron collider could be able to produce
such kind of particles, the ATLAS detector could be able to infer their presence
by the energy unbalance they would create. This thesis presents the result of
the search for physics beyond the Standard Model with the ATLAS detector in the
3.2~$\ifb$ delivered in 2015 in an experimental signature with jets and large
missing transverse momentum in the final state.
%%% Local Variables:
%%% mode: latex
%%% TeX-master: "../search_for_DM_LED_with_ATLAS"
%%% End:
