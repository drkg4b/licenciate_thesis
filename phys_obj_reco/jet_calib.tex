The \gls{jes} relates the energy measured by the ATLAS detector to the kinematic
properties of the corresponding stable particles, this calibration must take
into account the detector effects such as the non compensating nature of the
ATLAS hadronic calorimeter (see Section~\ref{sec:hadronic-shower}), energy loss
due to inactive regions in the detector or particle showers not fully contained
in the calorimeter. The \emph{electromagnetic scale} is the baseline signal
scale of the ATLAS calorimeters it is established during test beam measurements
with electrons and it accounts correctly for the energy deposited by
electromagnetic showers. The basic jet calibration scheme, applies JES
correction to the EM scale and is usually referred to as EM+JES. The \gls{lcw}
is another calibration method that clusters topologically connected cells,
classifying them as electromagnetic or hadronic and deriving the energy
corrections from single pion MC simulation and dedicated studies to account for
the detector effects, the jets calibrated with this method are referred to as
LCW+JES\@. The \gls{gcw} calibration uses the fact that electromagnetic and
hadronic showers leave a different energy deposition in the calorimeter cells,
with the electromagnetic shower more compact than the hadronic one. The energy
correction are then derived for each calorimeter cell within the jet and
minimizing the energy resolution. The \gls{gs} method starts from EM+JES
calibrated jets and corrects for the fluctuation in particle content of the
hadronic shower using the topology of the energy deposits. The corrections are
applied in a way that leaves unchanged the mean jet energy~\cite{JetCalib}.
%%% Local Variables:
%%% mode: latex
%%% TeX-master: "../search_for_DM_LED_with_ATLAS"
%%% End:
