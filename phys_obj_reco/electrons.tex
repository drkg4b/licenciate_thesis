Electrons are identified in the central part of the ATLAS detector
($|\eta| < 2.47$) by an energy deposit in the electromagnetic calorimeter and an
associated track in the inner detector. Signal electrons are defined as prompt
electrons coming from the decay of a $W, Z$ boson or a top quark while
background electrons come from hadronic jets, photon conversion and
semi-leptonic heavy flavor hadron decay. A likelihood discriminator is formed
using the shower shape in the EM calorimeter, the track-cluster matching, some
of the track quality distributions from signal and background simulation and
cuts on the number of hits in the ID. Cuts that depends on $|\eta|$ and $\et$ on
the likelihood estimator allow to distinguish between signal and background
electrons.

Electron identification efficiencies are measured in $pp$ collisions data and
compared to efficiencies measured in $\zee$ simulations. Signal electrons can
furthermore be selected with different sets of cuts for the likelihood-based
criteria with $\sim 95\%, \sim 90\%$ and $\sim 80\%$ efficiency for electrons
with $\pt \sim 40~$GeV. The different criteria are referred to as \emph{loose},
\emph{medium} and \emph{tight} operating points respectively\cite{ATL-EL-IDENT}
where, for example, a tight criterion lead to a higher purity of signal
electrons.

In this analysis, the \emph{baseline electrons} are selected requiring a
transverse energy $\et > 20$~GeV, $|\eta| < 2.47$, to satisfy the \emph{loose}
likelihood selection criteria. Furthermore, it is required that no dead EM
calorimeter front-end board (FEB) or high voltage (HV) channels in the
calorimeter cluster are present and that the baseline electron passes the
OR. The baseline electron criteria is used to veto electrons used in the muon
control regions and the signal region definition. In addition to all the
baseline criteria, the \emph{good electron} definition requires the electrons to
satisfy the \emph{tight} likelihood selection criteria, the electron track $d_0
/ \sigma_{d0} < 5$~mm and $|z_0| < 0.5$~mm and the \emph{LooseTrackOnly}
electron isolation criteria.

\begin{table}[!th]
  \centering
  \begin{tabular}{cc}
    \multicolumn{2}{c}{Electron Definition} \\
    \hline \hline \\
    \textbf{Baseline electron} & \textbf{Good electron} \\
    \hline
    $\et > 20$~GeV & \emph{baseline} \\
    $|\eta| < 2.47$ & \emph{tight} \\
    \emph{loose} & $d_0 / \sigma_{d0} < 5$~mm \\
    No dead FEB in the EM calo cluster & $|z_0| < 0.5$~mm \\
    No dead HV in the EM calor cluster & \emph{LooseTrackOnly} \\
    passes the OR & \\
    \hline \hline
  \end{tabular}
  \caption{Monojet electron definition}
  \label{tab:ele_def}
\end{table}
%%% Local Variables:
%%% mode: latex
%%% TeX-master: "../search_for_DM_LED_with_ATLAS"
%%% End:
