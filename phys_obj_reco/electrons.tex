Electrons are identified in the central part of the ATLAS detector
($|\eta| < 2.47$) by an energy deposit in the electromagnetic calorimeter and an
associated track in the inner detector. Signal electrons are defined as prompt
electrons coming from the decay of a $W, Z$ boson or a top quark while
background non--prompt electrons come from hadron decays, photon conversion,
semi-leptonic heavy flavor hadron decay and highly electromagnetic jets. A
likelihood discriminant is formed using different information: the shower shape
in the EM calorimeter, the track-cluster matching, some of the track quality
distributions from signal and background simulation and cuts on the number of
hits in the ID\@. Cuts that depends on $|\eta|$ and $\et$ on the likelihood
estimator allow to distinguish between signal and background electrons.

Electron identification efficiencies are measured in $pp$ collision data and
compared to efficiencies measured in $\zee$ simulations. Signal electrons can
furthermore be selected with different sets of cuts for the likelihood-based
criteria with $\sim 95\%, \sim 90\%$ and $\sim 80\%$ efficiency for electrons
with $\pt = 40~$GeV. The different criteria are referred to as \emph{loose},
\emph{medium} and \emph{tight} operating points respectively~\cite{ATL-EL-IDENT}
where, for example, the tight criterion leads to a higher purity of signal
electrons but has a lower efficiency than the looser criteria.

In this analysis, the \emph{baseline electrons} are selected requiring a
transverse energy $\et > 20$~GeV, $|\eta| < 2.47$, they need to satisfy the
\emph{loose} likelihood selection criteria, it is required that no dead EM
calorimeter \gls{feb} or \gls{hv} channels in the calorimeter cluster are
present and that the baseline electron survives the overlap removal with other
particles as described in \cref{sec:overlap-removal}. The baseline electron
criteria is used to veto electrons used in the muon control regions and the
signal region definition. In addition to all the baseline criteria, the
\emph{good electron} definition requires the electrons to satisfy the
\emph{tight} likelihood selection criteria, the electron track
$d_0 / \sigma_{d0} < 5$ and $|z_0| < 0.5$~mm and the \emph{LooseTrackOnly}
electron isolation criteria which is based only on tracks and is 99\% efficient.
\begin{table}[!th]
  \centering
  \begin{tabular}{ll}
    \toprule
    \multicolumn{2}{c}{Electron Definition} \\
    \midrule \midrule
    \textbf{Baseline electron} & \textbf{Good electron} \\
    \midrule
    $\et > 20$~GeV & \emph{baseline} \\
    $|\eta| < 2.47$ & \emph{tight} working point \\
    \emph{loose} working point & $d_0 / \sigma_{d0} < 5$ \\
    No dead FEB in the EM calo cluster & $|z_0| < 0.5$~mm \\
    No dead HV in the EM calo cluster & \emph{LooseTrackOnly} \\
    passes the overlap removal & \\
    \bottomrule
  \end{tabular}
  \caption{Electron definition for the monojet analysis.}
  \label{tab:ele_def}
\end{table}
%%% Local Variables:
%%% mode: latex
%%% TeX-master: "../search_for_DM_LED_with_ATLAS"
%%% End:
