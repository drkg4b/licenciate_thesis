During object reconstruction, it may happen that different algorithms identify
the same track and cluster as different types of particles, this results in a
duplicate object. In physics analyses a decision must be made on which
interpretation to give to the reconstructed object, this process is called
\gls{or}~\cite{Alison:1536507}.

In this analysis, an overlap removal is applied to electrons, muons and jets
that pass the baseline criteria and the following objects are removed:
\begin{itemize}
\item Remove jet in case any pair of jet and electron satisfies $\Delta R(j,
  e) < 0.2$.
\item Remove electron in case any pair of jet and electron satisfies $0.2 <
  \Delta R(j, e) < 0.4$.
\item Remove muon in case any pair of muon and jet with at least 3 tracks
  satisfies $\Delta R(j, \mu) < 0.4$.
\item Remove jet if any pair of muon and jet with less than 3 tracks satisfies
  $\Delta~R(j, \mu)~<~0.4$.
\end{itemize}
%%% Local Variables:
%%% mode: latex
%%% TeX-master: "../search_for_DM_LED_with_ATLAS"
%%% End:
