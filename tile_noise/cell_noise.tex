In most cases, cell noise is the combination of the pulse from the PMTs
connected to a cell (as mentioned the E layer cells are connected to only one
PMT). The digital noise from the PMTs is added quadratically and converted in
MeV using the calibration constants and eq.~\eqref{eq:70}, this results in four
gain combinations: \gls{hghg}, \gls{lglg}, \gls{lghg} and \gls{hglg}. The cell
noise is used to identify the seed cells in the topocluster algorithm (see
Section~\ref{sec:topocluster}).

Figure~\ref{fig:non_gaussianity} shows a comparison between the cell noise and
the fitted $\sigma$ parameter of a normal distribution. The ratio RMS / $\sigma$
= 1 indicates a perfect agreement between the measured and the fitted amplitude
distribution for a single Gaussian hypothesis; as can be seen, with an old model
of \gls{lvps} (blue square) the ratio RMS / $\sigma$ can be larger thus
worsening the performance of the topological clustering algorithm. For this
reason a double Gaussian distribution is used to fit the energy distribution,
the probability density function is defined as:
\begin{equation}
  \label{eq:71}
  f_{2g} = \frac{1}{1 + R} \left( \frac{1}{\sqrt{2 \pi} \sigma_1} e^{-
      \frac{x^2}{2 \sigma_1^2}} + \frac{R}{\sqrt{2 \pi} \sigma_2} e^{-
      \frac{x^2}{2 \sigma_2^2}} \right)
\end{equation}
where $R$ is the relative normalization of the two Gaussians and
$\sigma_1, \sigma_2$ and $R$ are independent parameters. These three are used to
define the region $\sigma_{\text{eff}}(E)$ where the significance for the double
Gaussian is the same as the one $\sigma$ region for a single Gaussian,
i.e.
$\int_{- \sigma_{\text{eff}}}^{\sigma_{\text{eff}}} f_{2g} =
0.68$~\cite{TileReadiness}.
In terms of $\sigma_{\text{eff}}$, for an energy deposit $E$, the significance
can be expressed as:
\begin{equation}
  \label{eq:72}
  \frac{E}{\sigma_{\text{eff}}(E)} = \sqrt{2}\ \text{Erf}^{- 1} \left( \frac{\sigma_1
      \text{Erf} \left(\frac{E}{\sqrt{2 \sigma_1}} \right) + R \sigma_2 \text{Erf}
    \left( \frac{E}{\sqrt{2 \sigma_2}} \right)}{\sigma_1 + R \sigma_2} \right)
\end{equation}
where Erf is the error function. Eq.~\eqref{eq:72} is the input to the
clustering algorithm, moreover this definition allows to use the same unit to
describe the noise for both the TileCal and LAr calorimeters. The region
$\sigma_{\text{eff}}(E)$ is commonly referred to as \emph{cell noise} and
together with the three double Gaussian parameters ($\sigma_1$, $\sigma_2$ and
R) is stored in the COOL database.

\begin{figure}[!h]
  \centering
    \includegraphics[width=.7\linewidth]{non_gaussianity}
    \caption{Comparison between the TileCal electronic noise, measured as the
      RMS of the reconstructed amplitude distribution in pedestal runs and the
      $\sigma$ of the Gaussian fit of the distribution for the old and new
      LVPS~\cite{TileCalNoisePub}.}
    \label{fig:non_gaussianity}
\end{figure}
%%% Local Variables:
%%% mode: latex
%%% TeX-master: "../search_for_DM_LED_with_ATLAS"
%%% End:
