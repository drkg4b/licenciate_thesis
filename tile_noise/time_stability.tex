The calibration constants used in \cref{eq:70}, as mentioned in
\cref{sec:tilecal-calibration}, are determined by different subsystems during
regular dedicated calibration runs. Each calibration constant is valid over a
period of time called \gls{iov}. The cell noise can vary due to a change in the
calibration, the digital noise or the channel status in a particular run. This
change must be consistent and checked.

The different TileCal subsystems (laser, CIS, etc.) all use a common software
framework, \gls{tucs}, to perform validity checks on a number of different
studies. To test the stability over time of the updated noise constants, a set
of python scripts was developed to expand the TUCS functionality. These scripts
allow to visually display the relative change of the cell noise and digital
noise constants, the channel status and the ratio between the cell noise and a
variable called RMS$_\text{eff}$ and defined as:
\begin{equation}
  \label{eq:73}
  \text{RMS}_{\text{eff}} = \sqrt{(1 - R) \sigma_1^2 + R \sigma_2^2}
\end{equation}
where $\sigma_1$, $\sigma_2$ and R are the free parameter in the double Gaussian
model (see Section~\ref{sec:cell-noise}).  The ratio $\sigma$ /
RMS$_\text{eff}$, where $\sigma$ is the cell noise, can be used to test the
goodness of the double Gaussian model: if $\sigma$ / RMS$_\text{eff}$ equals
one, the double Gaussian well models the noise, if $\sigma$ / RMS$_\text{eff}$
$> 1$, it means that there is noise that is not well described by it.

Figure~\ref{fig:jumps} shows the time evolution plot for two representative
TileCal cells. In Figure~\ref{fig:no_jumps} it can be seen that cell number 2 in
the BC layer (BC2) of the 41st module in the C side of LB (LBC 41) is stable
over several pedestal runs. In Figure~\ref{fig:with_jumps} on the other hand, it
is possible to see a variation in the cell noise and, accordingly, of the
$\sigma$ / RMS$_{\text{eff}}$ without a compatible variation in the calibration,
in the digital noise constants or in the channel status. The term \emph{jump}
will be used in the following to indicate a variation in the cell noise not
compatible with a change in the other quantities.

\begin{figure}[!h]
  \centering
  \begin{subfigure}[t]{.7\linewidth}
    \includegraphics[width=\linewidth]{no_jumps}
    \caption{Control cell with no variation.}
    \label{fig:no_jumps}
  \end{subfigure}

  \begin{subfigure}[t]{.7\linewidth}
    \includegraphics[width=\linewidth]{with_jumps}
    \caption{Cell with a variation.}
    \label{fig:with_jumps}
  \end{subfigure}
  \caption{Time evolution plots for two different representative cells in the
    calorimeter. The plot shows the change relative to the first run considered
    of several quantities for different IOVs (vertical dashed lines). If a
    channel is of due to some problems (Bad channel), this is reported in the
    plot with a black square.}
  \label{fig:jumps}
\end{figure}

This problem was investigated by writing a software to manually fit the pulse
shape and calculate the noise constants focusing on two specific IOVs,
$[183110, 183382[$ and $[183382, 183515[$. Some calorimeter cells without jump
were used to validate the noise constants calculated with the fit and those
stored in the COOL database. Figure~\ref{fig:no_jump_fit} shows the control cell
energy distribution with the double Gaussian fit superimposed for two runs where
the jump was present in other cells. The results for the ratio of the amplitude
of the double Gaussian model (R), the RMS$_{\text{eff}}$ and the $\sigma$ /
RMS$_{\text{eff}}$ obtained with the fit are:
\begin{equation}
  \begin{cases}
    R : 0.0003 \\
    \text{RMS}_{\text{eff}}: 20.12 \\
    \sigma / \text{RMS}_{\text{eff}}: 0.998
  \end{cases}
  \to
  \begin{cases}
    R : 0.0002 \\
    \text{RMS}_{\text{eff}} : 20.07 \\
    \sigma / \text{RMS}_{\text{eff}}: 0.995.
  \end{cases}
  \label{eq:74}
\end{equation}
They are in good agreement with the values stored in the COOL database and
reported in Table~\ref{tab:no_jump_fit}.

\begin{figure}[!h]
  \centering
  \begin{subfigure}[t]{.7\linewidth}
    \includegraphics[width=\linewidth]{no_jump_fit_before}
    \caption{Fit before jump.}
    \label{fig:no_jump_fit_before}
  \end{subfigure}

  \begin{subfigure}[t]{.7\linewidth}
    \includegraphics[width=\linewidth]{no_jump_fit_after}
    \caption{Fit after jump.}
    \label{fig:no_jump_fit_after}
  \end{subfigure}
  \caption{Fit of the reconstructed pulse shape on a control cell with no
    variation (jump) in the cell noise.}
  \label{fig:no_jump_fit}
\end{figure}

\begin{table}[!h]
  \centering
  \begin{tabular}{r c}
    \toprule
    \multicolumn{2}{c}{LBC41 BC2 Values From Database} \\
    \midrule \midrule
    \multicolumn{2}{c}{Before Jump} \\
    \midrule
    $\sigma$: & 20.08 \\
    $\sigma_1$: & 19.97 \\
    $\sigma_2$: & 80.59 \\
    R\@: & 0.00026 \\
    RMS$_\text{eff}$: & 20.01 \\
    $\sigma$ / RMS$_\text{eff}$: & 1.0035 \\
    \bottomrule
  \end{tabular} \quad
  \begin{tabular}{r c}
    \toprule
    \multicolumn{2}{c}{LBC41 BC2 Values From Database} \\
    \midrule \midrule
    \multicolumn{2}{c}{After Jump} \\
    \midrule
    $\sigma$: & 19.98 \\
    $\sigma_1$: & 19.94 \\
    $\sigma_2$: & 71.41 \\
    R\@: & 0.00023 \\
    RMS$_\text{eff}$: & 19.97 \\
    $\sigma$ / RMS$_\text{eff}$: & 1.0006 \\
    \bottomrule
  \end{tabular}
  \caption{The table reports the cell noise constants stored in the COOL
    database for two different run numbers corresponding to before and after the
  jump for a cell where there is no variation in the cell noise.}
\label{tab:no_jump_fit}
\end{table}

Figure~\ref{fig:jump_fit} shows the energy distribution with the double Gaussian
fit superimposed on the seventh cell of the BC layer (BC7) on the C side of the
LB partition of the 41st module (LBC 41). The cell had the jump under
investigation (see Figure~\ref{fig:jumps}) and this is reflected in the fit results:
\begin{equation}
  \label{eq:75}
  \begin{cases}
    R : 0.042 \\
    \text{RMS}_{\text{eff}}: 29.59 \\
    \sigma / \text{RMS}_{\text{eff}}: 1.37
  \end{cases}
  \to
  \begin{cases}
    R : 0.014 \\
    \text{RMS}_\text{{eff}} : 26.67 \\
    \sigma / \text{RMS}_{\text{eff}}: 1.2.
  \end{cases}
\end{equation}
Also in this case, the noise constants from the fit, are in agreement with those
stored in the COOL database and reported in  Table~\ref{tab:jump_fit}. Moreover,
the $\chi^2$ of the distribution and the ration $\sigma$ / RMS$_{\text{eff}}$
greater than one, imply that the double Gaussian model is not a good model in
this case.

\begin{figure}[!h]
  \centering
  \begin{subfigure}[t]{.7\linewidth}
    \includegraphics[width=\linewidth]{jump_fit_before}
    \caption{Before jump.}
    \label{fig:jump_fit_before}
  \end{subfigure}

  \begin{subfigure}[t]{.7\linewidth}
    \includegraphics[width=\linewidth]{jump_fit_after}
    \caption{After jump.}
    \label{fig:jump_fit_after}
  \end{subfigure}
  \caption{Fit of the reconstructed pulse shape on a cell with variation (jump)
    in the cell noise non compatible with a change in the calibration, digital
    noise or channel status.}
  \label{fig:jump_fit}
\end{figure}

\begin{table}[!h]
  \centering
  \begin{tabular}{r c}
    \toprule
    \multicolumn{2}{c}{LBC41 BC7 Values From Database} \\
    \midrule \midrule
    \multicolumn{2}{c}{Before Jump} \\
    \midrule
    $\sigma$: & 42.19 \\
    $\sigma_1$: & 24.26 \\
    $\sigma_2$: & 99.16 \\
    R\@: & 0.037 \\
    RMS$_{\text{eff}}$: & 30.56 \\
    $\sigma$ / RMS$_{\text{eff}}$: & 1.38 \\
    \bottomrule
  \end{tabular} \quad
  \begin{tabular}{r c}
    \toprule
    \multicolumn{2}{c}{LBC41 BC7 Values From Database} \\
    \midrule \midrule
    \multicolumn{2}{c}{After Jump} \\
    \midrule
    $\sigma$: & 32.12 \\
    $\sigma_1$: & 24.42 \\
    $\sigma_2$: & 94.56 \\
    R\@: & 0.014 \\
    RMS$_{\text{eff}}$: & 26.79 \\
    $\sigma$ / RMS$_{\text{eff}}$: & 1.20 \\
    \bottomrule
  \end{tabular}
  \caption{The table reports the cell noise constants stored in the COOL
    database for two different run numbers corresponding to before and after the
  jump for a cell where there is a variation in the cell noise was spotted.}
  \label{tab:jump_fit}
\end{table}

After consultation with experts~\cite{PrivateConv}, it was suggested that this
behavior could be caused by the \gls{tnf}. Many electronic devices are involved
in the signal reconstruction, the noise of these can be altered in a coherent
way (by electromagnetic field emission for instance) and thus altering the jet
and missing transverse energy reconstruction. This alteration is called
\emph{coherent noise}. The TNF calculates for each motherboard the pedestal
average:
\begin{equation}
  \label{eq:92}
  d = \frac{\sum_i^N d_i}{N}
\end{equation}
where $d_i$ is the $i$--th channel data in ADC counts and $N$ is the number of
channels. This average can be regarded as an estimation of the coherent noise
and is subtracted from the channel data ($d_i - d$) on an event--by--event
basis.

The cell noise was recalculated without noise filter and the corresponding
distribution re-fitted obtaining:
\begin{equation}
  \label{eq:76}
  \begin{cases}
    R : 0.0750867 \\
    \text{RMS}_\text{eff} : 47.49 \\
    \sigma / \text{RMS}_\text{eff}: 1.49
  \end{cases}
  \to
  \begin{cases}
    R : 0.075  \\
    \text{RMS}_\text{eff}: 48.91 \\
    \sigma / \text{RMS}_\text{eff}: 1.44.
  \end{cases}
\end{equation}
Comparing Equation~\ref{eq:76} with Equation~\ref{eq:75} it is possible to see
that without TNF, there is no jump. \cref{fig:no_filter_fit} show the double
Gaussian fit applied to another cell where the jump was present,
in \cref{fig:d2_fit_before,fig:d2_fit_after} the noise filter is on while in
\cref{fig:d2_fit_before_no_flt,fig:d2_fit_after_no_flt} the TNF was removed. It
can be seen that the fit improves when there is no noise filter.

\begin{figure}[!h]
  \centering
  \begin{subfigure}[t]{.48\linewidth}
    \includegraphics[width=\linewidth]{d2_fit_before}
    \caption{Before jump.}
    \label{fig:d2_fit_before}
  \end{subfigure}
  \begin{subfigure}[t]{.48\linewidth}
    \includegraphics[width=\linewidth]{d2_fit_after}
    \caption{After jump.}
    \label{fig:d2_fit_after}
  \end{subfigure}

  \begin{subfigure}[t]{.48\linewidth}
    \includegraphics[width=\linewidth]{d2_fit_before_no_flt}
    \caption{Before jump.}
    \label{fig:d2_fit_before_no_flt}
  \end{subfigure}
  \begin{subfigure}[t]{.48\linewidth}
    \includegraphics[width=\linewidth]{d2_fit_after_no_flt}
    \caption{After jump.}
    \label{fig:d2_fit_after_no_flt}
  \end{subfigure}
  \caption{Comparison of the reconstructed pulse shape of a cell with the cell
    noise variation without a corresponding variation in he calibration
    constants, digital noise or channel status (jump) with the double Gaussian
    fit superimposed with the noise filter
    (\cref{fig:d2_fit_before,fig:d2_fit_after}) and without noise filter
    (\cref{fig:d2_fit_before_no_flt,fig:d2_fit_after_no_flt}).}
  \label{fig:no_filter_fit}
\end{figure}
%%% Local Variables:
%%% mode: latex
%%% TeX-master: "../search_for_DM_LED_with_ATLAS"
%%% End:
