The energy resolution of a detector measures its ability of distinguishing
between particles of different energies; the better the energy resolution, the
better it can separate energy peaks belonging to different decays.

The energy resolution can be written as:
\begin{equation}
  \label{eq:64}
  \frac{\sigma_E}{E} = \frac{a}{\sqrt{E}} \oplus \frac{b}{E} \oplus c,
\end{equation}
where the $\oplus$ symbol indicates a quadratic sum. The first term ($a$) in the
equation is the \emph{stocastic term}, it is mainly due to fluctuations related
to the physical evolution of the shower. In homogeneous calorimeters, this term
is small because the energy deposited in the active volume by a monochromatic
beam of particles is constant for each event. In a sampling calorimeter, the
active layers are interleaved with absorber layers thus the energy deposited in
the active material fluctuates event by event. These are called \emph{sampling
  fluctuations} and, in sampling electromagnetic calorimeters, represents the
greatest limitation to energy resolution due to the variation in the number of
charged particles which cross the active layers. The second term ($b$) in
eq.~\eqref{eq:64} is called the \emph{noise term}, it comes from the electronic
noise of the detector readout chain. Sampling or homogeneous calorimeters which
collect the signal in the form of light, using for example a photo-multiplier
tube with a high gain multiplication of the signal with a low electronic noise,
can achieve low levels of noise. Calorimeters that collect the signal in form of
charge, must use an pre-amplifier having thus a higher level of noise. In
sampling calorimeters, the noise term can be further reduced by increasing the
sampling fraction, this way there is a larger signal coming from the active
material and a higher noise--to--signal ratio. The last term ($c$) of the
equation is the \emph{constant term}, it does not depend on the energy of the
particles but includes all the non uniformities in the detector response such as
instrumental effect, imperfections in the calibration of different parts of the
detector, radiation damage, detector aging, or the detector
geometry~\cite{Calorimetry}.
%%% Local Variables:
%%% mode: latex
%%% TeX-master: "../search_for_DM_LED_with_ATLAS"
%%% End:
