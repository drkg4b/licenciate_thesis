Electrons and photons can lose energy by \emph{ionization} or
\emph{radiation}. When electrons with energies greater than $\sim 10$~MeV
interacts with the field generated by the absorber layer nuclei, a deceleration
and thus a loss of energy in form of a photon (\emph{Bremsstrahlung}) is
experienced. Photons in this energy range produce mostly electron-positron
pairs. At lower energies, electrons lose their energy through ionization and
thermal excitation of the active material atoms while photons lose energy
through the Compton scattering and the photoelectric effect.

Electrons and photons with a sufficient amount of energy interacting with an
absorber, produce secondary photons through Bremsstrahlung or secondary
electrons and positrons by pair production. These secondary particles will
produce more particles through the same mechanisms giving rise to a shower of
particles with progressively lower energies. This process goes on until the
energy of the electrons falls below a critical energy, $\epsilon$, where
ionization and excitation becomes the dominant effects~\cite{Calorimetry}.
%%% Local Variables:
%%% mode: latex
%%% TeX-master: "../search_for_DM_LED_with_ATLAS"
%%% End:
