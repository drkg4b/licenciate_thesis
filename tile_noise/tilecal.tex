\textbf{I'm not very happy with this section, will have a review!}
TileCal is the central hadronic calorimeter of the ATLAS experiment, it is
designed for energy reconstruction of hadrons, jets, tau particles and missing
transverse energy. TileCal is a scintillator steel non compensated sampling
calorimeter and it covers the region of $|\eta| < 1.7$. The scintillation light
produced in the tiles is transmitted by wavelength shifting fibers to
\glspl{pmt}. The analog signals from the PMTs are amplified, shaped and
digitized by sampling the signal every 25~ns. The TileCal front end electronics
read out the signals produced by about 10000 channels measuring energies ranging
from 30~MeV to 2~TeV. The readout system is responsible for reconstructing the
data in real time. The digitized signals are reconstructed with the Optimal
Filtering algorithm, which computes for each channel the signal amplitude, time
and quality factor at the required high rate.

TileCal is designed as one \gls{lb} and two \gls{eb}. The barrels are further
divided, according to their geometrical position on the $z$-axis, in partitions
called EBA, LBA, EBC and LBC (see Section~\ref{sec:coordinate-system}). Each
partition consists of 64 independent wedges (see Figure~\ref{fig:tile_mod})
along the azimuthal direction called \emph{modules}; the LBA and EBA partitions
are shown in Figure~\ref{fig:tile_cells}.

\begin{figure}[!h]
  \centering
    \includegraphics[width=.3\linewidth]{tile_module}
    \caption{Cut away showing the optical read out and design of a TileCal
      module.}
    \label{fig:tile_mod}
\end{figure}

Between the LB and the EB there is a 600~mm gap needed for the ID and the LAr
cables, electronics and services. Part of the gap contains the \gls{itc}, a
detector designed to maximize the active material while leaving enough space for
services and cables. The ITC is an extension of the EB and it occupies the 0.8 <
$|\eta|$ < 1.6 region. The combined 0.8 < $|\eta|$ < 1.0 part is called
\emph{plug} and in the 1.0 < $|\eta|$ < 1.6 region, due to the very limited
space, the ITC is composed of only scintillator. The scintillators between 1.0 <
$|\eta|$ < 1.2 are called \emph{gap scintillators}, while those between 1.2 <
$|\eta|$ < 1.6 are called \emph{crack scintillators}. The plug and the gap
scintillators mainly provide hadronic shower sampling while the crack
scintillator, which extends to the region between the barrel and the end-cap
cryostats, samples the electromagnetic shower in a region where the normal
sampling is impossible due to the dead material of the cryostat walls and the ID
cables.

TileCal is also divided in longitudinal layers, the A, BC and D layers as shown
in Figure~\ref{fig:tile_cells}; the two innermost layers have a
$\Delta \eta \times \Delta \phi$ segmentation of $0.1 \times 0.1$ while in the
outermost, the segmentation is $0.1 \times 0.2$. Each layer is logically divided
into \emph{cells} by grouping together in the same PMT the fibers coming from
different scintillators belonging to the same radial depth. The gap/crack
scintillators are also called E layer cells.

The energy resolution for jets of TileCal is:
\begin{equation}
  \label{eq:65}
  \frac{\sigma_E}{E} = \frac{50\%}{\sqrt{E}} \oplus 3\%
\end{equation}
for $|\eta| < 3$. The 3\% constant term becomes dominant for high energy hadrons
where an increase in energy resolution is expected~\cite{TileCal}.

\begin{figure}[!h]
  \centering
    \includegraphics[width=\linewidth]{tile_cells}
    \caption{}
    \label{fig:tile_cells}
\end{figure}
%%% Local Variables:
%%% mode: latex
%%% TeX-master: "../search_for_DM_LED_with_ATLAS"
%%% End:
