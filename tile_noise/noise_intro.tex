TileCal periodically records sets of events (referred to as \emph{runs}) with no
signal in the PMTs, called \emph{pedestal runs}. During these runs, each channel
is read out using both, HG and LG for about 100000 events. These events are
sampled every 25~ns in 7 samples as in normal physic runs and are normally
distributed around a mean value called \emph{pedestal}. The \gls{rms} of the
pedestal is the \emph{noise}. Pedestal runs are used to calculate different
parameters called \emph{noise constants} that allow to describe the electronic
noise. Two different sets of noise constants are computed: \emph{Digital Noise}
(or \emph{Sample Noise}) and \emph{Cell Noise}.
%%% Local Variables:
%%% mode: latex
%%% TeX-master: "../search_for_DM_LED_with_ATLAS"
%%% End:
