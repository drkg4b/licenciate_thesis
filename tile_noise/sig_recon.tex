The TileCal cells are read out by two PMTs with the exception of the E layer
cells that are connected to only one photomultiplier tube using \gls{wsf}. Each
PMT is associated to an electronic read-out channel. The current pulse from the
PMTs is shaped and amplified by the 3-in-1 card. There are two possible gains:
\gls{hg} and \gls{lg}, with an amplification ratio of 64. The 3-in-1 card forms
the front-end electronics of the read-out chain and provides three basic
functions: shaping of the pulse, charge injection calibration and slow
integration of the PMT signals for monitoring and calibration~\cite{TileCal}. Up
to twelve 3-in-1 cards are serviced by a motherboard that provides power and
individual control signals.  The amplified signal is sent to two \glspl{adc}
synchronous with the 40~MHz LHC clock thus sampling the signal every 25~ns. For
optimization and efficiency reasons, 7 samples for each pulse are taken and sent
to the \glspl{rod} for a L1 accept.
%%% Local Variables:
%%% mode: latex
%%% TeX-master: "../search_for_DM_LED_with_ATLAS"
%%% End:
